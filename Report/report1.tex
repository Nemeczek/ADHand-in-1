\documentclass{tufte-handout}
\usepackage{amsmath}
\usepackage[utf8]{inputenc}
\usepackage{mathpazo}
\usepackage{booktabs}
\usepackage{microtype}

\pagestyle{empty}


\title{Stable Matching Report}
\author{Alice Cooper and Bob Marley}

\begin{document}
  \maketitle

  \section{Results}

  Our implementation produces the expected results on all input--output file pairs, except {\tt sm-random-100.txt}, where it matches 54 with 12 instead of 2.
  We have no idea why this happens.%
  \sidenote{%
  Complete the report by filling in your correct names,
  filling in the parts marked $[\ldots]$,
  and changing other parts wherever necessary.
  For instance, if your implementation passes all tests, then write that.
  Remove the sidenotes in your final hand-in.
  }

  On input {\tt sm-bbt-in.txt}, we produce the following matching:
  \begin{quotation}
    Sheldon--Amy, Rajesh--Penny, Howard--Bernadette, Leonard--Priya.  \sidenote{Replace with your results.}
  \end{quotation}

  \section{Implementation details}
  
  We start by loading the specific file, by copying all the file content into a String. Hereafter we split this by \textit{newlines} into a String array.
  Next we parse the string array, by using our \text{Transformer}class. This data is saved into our datastructures, which is Integer and
  String arrays and multidimensional String arrays. We then create a stack which we fill with un-matched men. 
  We start iterating through the stack of un-matched men, and basically just follow the pseudo implementation of the Gale Shapley algorithm \sidenote{Introduction and Stable Matching slides p. 9}

  The men's preferences are stored in a left-leaning red--black B-tree with $B=12$ and no inverse anti-snails,%
  \sidenote{Replace with whatever data structure you actually use.}
   as described in Section 5.2 of Kleinberg and Tardos, \emph{Algorithms Design}, Addison--Wesley 2013.%
   \sidenote{If you refer to something, like a book or a stackexchange debate, or an external library, be professional about it.}
  The women's preferences are stored in a heap.\sidenote{Replace with whatever data structure you actually use.}

  We can check find a free man who has not proposed to every woman in time $[\ldots]$,
  because we store $[\ldots]$.
  
  The FileParser and Transformer classes both have complexity of O(n). In our Gale Shapley implementation, we run some \textit{for} loops inside of the \textit{while} loop,
  which both runs at O(n). Since they are nested, the complexity of our implementation is O(n2). 

  With these data structures, our implementation runs in time $O(n\log n)$  \sidenote{Replace with your actual running time.}
  on inputs with $n$ men and $n$ women.


\end{document}
