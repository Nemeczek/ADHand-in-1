\documentclass{tufte-handout}
\usepackage{amsmath}
\usepackage[utf8]{inputenc}
\usepackage{mathpazo}
\usepackage{booktabs}
\usepackage{microtype}

\pagestyle{empty}


\title{Stable Matching Report}

\author{Mark Dear, Stelios Papaoikonomou, Radek Niemczyk, Rasmus Løbner Christensen}

\begin{document}
  \maketitle

  \section{Results}
  
  Our implementation produces the expected results on all input--output files. 
 
  For instance, on input {\tt sm-bbt-in.txt}, we produce the following matching:
  \begin{quotation}
    Sheldon--Amy, Rajesh--Penny, Howard--Bernadette, Leonard--Priya. 
  \end{quotation}

  \section{Implementation details}
  
  We start by loading the specific file, by copying all the file content into a String. Hereafter we split this by \textit{newlines} into a String array.
  Next we parse the string array, by using our \text{Transformer}class. This data is saved into our datastructures, which is Integer and
  String arrays and multidimensional String arrays. We then create a stack which we fill with un-matched men. 
  We start iterating through the stack of un-matched men, and basically just follow the pseudo implementation of the Gale Shapley algorithm \sidenote{Introduction and Stable Matching slides p. 9}
  
  The FileParser and Transformer classes both have complexity of O(n). In our Gale Shapley implementation, we run some \textit{for} loops inside of the \textit{while} loop,
  which both runs at O(n). Since they are nested, the complexity of our implementation is \sqrt[2]{n}. 

\end{document}
